\documentclass[11pt, twoside, a4paper]{article}
\usepackage{color}
\usepackage{tikz}
\usepackage{listings}
\usetikzlibrary{shapes, arrows}
\tikzstyle{decision} = [ diamond, aspect=2, draw, fill=blue!20, text width=5em, text badly centered, node distance=3cm, inner sep=0pt ]
\tikzstyle{block} = [ rectangle, draw, fill=blue!20, text width=5em, text centered, rounded corners, minimum height=4em ]
\tikzstyle{line} = [ draw, -latex' ]
\tikzstyle{cloud} = [draw, ellipse,fill=red!20, node distance=3cm,
    minimum height=2em]


\begin{document}
\title{General responsibilities of a node}
\section {General responsibilities of a node}
Common responsibilities of a node.
\begin{itemize}
  \item Some data types
  \begin{lstlisting}[language=Haskell]
      -- data WhereIsReply = WhereIsReply String (Maybe ProcessId)
  \end{lstlisting}

  \item Initialize Node 
    \begin{lstlisting}[language=Haskell]
      -- getall peers other than me 
      -- register my pid as a list of available services.
      -- listen to all peers to copy the global state into the node.
    \end{lstlisting}
  \item Handle WhereIs Reply
    \begin{lstlisting}[language=Haskell]
        -- handle WhereIsReply
        -- Throw exception when ProcessId is Nothing.
        handleWhereIsReply :: Server -> ServiceProfile -> WhereIsReply -> Process ()
    \end{lstlisting}
  \item Broadcast local message 
    \begin{lstlisting}[language=Haskell]
      -- Broadcast local message by publishing to a local STM Bounded queue.
      -- bMessage :: (WS.Connection, TBQueue LocalMessage) -> (Text -> ProcessId -> LocalMessage) -> BroadcastMessageT LocalMessage
    \end{lstlisting}
  \item Broadcast remote messages to local client. 
    \begin{lstlisting}[language=Haskell]
      -- sendWelcomeMessages :: (ProcessId, MessageId, Server) -> IO ()
    \end{lstlisting}
  \item Update remote service queue
    \begin{lstlisting}[language=Haskell]
      updateRemoteServiceQueue :: Server -> ProcessId -> (PMessage, UTCTime) -> STM ProcessId
    \end{lstlisting}

\end{itemize}


\end{document}